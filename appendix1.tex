\documentclass[conference]{IEEEtran}
\usepackage{textcomp}
\usepackage{lscape}
\usepackage{graphicx}
\usepackage{cite}
\usepackage{amsmath}
\usepackage{gensymb}

\begin{document}
\twocolumn[
\centering
\LARGE Appendix 1: ELEN4002 Group Reflection
\linebreak\linebreak
\normalsize Darrion Singh (1056673)
]

% ** insert table

The division of work is shown in Table \ref{division} above.
The author of this reflection is hereafter referred to as the 'author', whereas the second group member, Sachin Govender (1036148) is hereafter referred to as the 'partner'.
While the work was split, both group members separately came up with total system solutions during project planning.
Afterwards, the solutions were presented to each other, and each member noted the strengths and weaknesses of the approach.
Finally, the strengths of each solution proposal were amalgamated to produce the final project plan, the outcome of which can be seen in Appendix 3.

Deciding upon the division of work was simple, as both members revealed their particular strengths and prior experience.
The author, having seen and experienced basic Image Processing during his Biomedical undergraduate years, as well as having seen and made basic use of Python -- the technology decided upon by both members -- volunteered to work on the Extraction layer.
Since the Extraction layer required to be functional in order to produce data for the Machine Learning layer, it was hoped that the authors prior experience in both areas of Image Processing and Python would result in the shortest possible time to generate training data.
The partner, having no experience with Python beforehand volunteered to learn both Python and the machine learning libraries during the time in which the author completed the Extraction layer, such that when the author finished data generation, the Machine Learning layer training scripts were already written and ready to produce models.

Upon meeting to work on the project, both group members started the working session by telling the other what task they were working on, an approximate time frame as to when their current task would be completed, and what they hoped to accomplish by the end of the day.
During these updates, each member had the opportunity to question the other on the reasons for their task, therein learning the processes involved for tasks and components that were not in their division of work.

During the week, both members met at the Fourth Year study on the first floor of the Chamber of Mines building to work.
During most weekends, working sessions took place at the residence of the partner, who was extremely hospitable and flexible with the long working hours that occurred at his residence.
At all times during the undertaking of this project, there was a mixture of encouragement, assistance, and productive argument regarding design processes and implementations.
This thorough work checking between the author and the partner occurred in an attempt to produce work of excellent quality.

There was open communication of problems that were encountered, often resulting in both members providing full attention until a viable solution or alternative was decided upon.
This was done as to prevent the project from going off-schedule from that established in the project plan, seen in Appendix 3.

As planned, when the author completed data generation, the partner had already gained a functional understanding of Python and had already written the model training scripts. commenced with training the models.
During this time, the author proceeded with code cleanup and documentation and creating the GUI.

Throughout the undertaking of this project, both members very easily made concessions during times when one could not make a working session, and an alternative day was quickly arranged.
At no point during the project did the author have any fear that the partner would not finish in time.
This kind of trustworthiness has proven to be invaluable to any project undertaking.

This experience has highlighted the importance of having clear lines of communication during a project, particularly during the beginning stages where falling behind schedule is all too easy.
It also highlighted the importance of teamwork during planning, as the solutions that either of us presented prior to the start of the project would not have been successful, as there existed major design flaws in both solutions.
Whilst this project experience was already enjoyable due to the software nature as well as the collaboration with a research clinic, it was even more enjoyable due to the lack of precarious positions one might find themselves in when completing complex projects.
This is a direct result of the effortless teamwork between the author and partner. trustworthy.
\end{document}